\documentclass[a4paper,12pt]{article}

\input{packages}

\begin{document}

\abstract{Dans ce document, nous abordons la base de la géométrie euclidienne: des axiomes aux théorèmes.}

\tableofcontents

\pagebreak
\section{Introduction}
Dans ce document, nous allons aborder les bases de la géométrie euclidienne en présentant les axiomes et certains théorèmes. Le mot géométrie provient du grec \textit{géo} - la terre et de \textit{métrie} - la mesure, à l'origine ce serait donc la mesure de ce qui fait partie de la terre.  La géométrie euclidienne étudie les figures et les mesures dans le plan et l'espace. A la base de cette géométrie se trouvent cinq axiomes et des définitions, qui sont tout ce que l'on a pour démontrer de plus en plus d'éléments, qui peuvent nous paraître triviaux tant ils sont omniprésents, mais qu'il a fallu démontrer.

\end{document}