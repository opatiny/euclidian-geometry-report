Segment: Portion d'une droite\\
Milieu: Point à égale distance des deux extrémités d'un segment\\
Sécant: Qui se croise en un point unique\\
Angle droit: Quart d'un tour\\
Triangle: Réunion de trois point non alignés\\
Cercle: 
Parallèles: Deux droites qui ne se croisent qu'à l'infini, c'est à dire qui ne se croisent pas.\\
Perpendiculaires: Deux droites séparées par un angle égal à un quart de tour\\
Médiatrice: Droite perpendiculaire à un segment et qui partage celui-ci en deux parties isométriques\\
Bissectrice: Ensemble des points équidistants à deux segments formant un angle, ou à deux droites sécantes\\
Médiane: Droite passant par le sommet d'un triangle et qui passe par le milieu du côté qui y est opposé\\
Hauteur: Droite passant pas l'un des sommets d'un triangle et qui est perpendiculaire au côté qui y est opposé\\
Triangle isocèle: Triangle ayant deux côtés et deux angles isométriques\\
Triangle équilatéral: Triangle dont les trois angles et les trois côtés sont isométriques\\
Triangle rectangle: Triangle dont l'un des angles est droit





\subsection{Théorèmes des médiatrices}
\begin{theorem}
La médiatrice de $AB$ est le lieu géométrique des points équidistants de $A$ et de $B$.
\end{theorem}
\begin{theorem}
Les médiatrices d'un triangle se croisent en un point unique.
\end{theorem}

