\documentclass[a4paper,12pt]{article}

\input{packages}

\begin{document}

\pagebreak
\subsection{Théorème de la transversale}
\begin{theorem}\label{th:transversale}
Si deux droites $d$ et $d'$ ayant $e$ comme transversale ont une paire d'angles alternes-internes, alternes-externes ou correspondant isométriques, alors ces deux droites sont isométriques.\\
Aussi, si l'on considère la transversale $e$ à deux droites $d$ et $d'$ parallèles, alors une paire d'angle est isométrique lorsque ces angles sont: alternes-internes, alternes-externes ou  correspondants.
\end{theorem}
\begin{proof}
Nous considérons deux droites $d$ et $d'$ qui on $e$ comme transversale.
\begin{figure}[H]
    \centering
    \includegraphics[scale=1]{theorems/transversal/Transversale.PNG}
\end{figure}


\begin{hyp}
     $d$, $d'$ et $e$ sont trois droites,
     $\alpha$ et $\alpha'$ sont correspondants et
     $\alpha \equiv \alpha'$
 \end{hyp}
 \begin{concl}
     $d \parallel d'$
 \end{concl}
 En considérant l'hypothèse, il existe deux cas possibles:
 \begin{enumerate}
     \item $d$ et $d'$ se coupent en un point r
     \item $d$ et $d'$ sont parallèles
 \end{enumerate}
 Il nous faut démontrer que le cas 1) est faux et que le deuxième cas est le seul possible.\\
 Supposons que $d$ et $d'$ se coupent en un point r, il y a alors deux cas de figure possibles:
 \begin{enumerate}[label=\emph{\alph*}.]
  \item $\alpha'$ est à l'intérieur du triangle $\triangle pp'r$ et $\alpha$ est un angle externe. Par conséquent, grâce au théorème de l'angle externe, on sait que $\alpha>\alpha'$, ce qui est absurde. Donc il est impossible que d et d' se croisent et que $\alpha$ ou $\alpha'$ soient à l'intérieur du triangle $\triangle pp'r$.
  
    \begin{figure}[H]
        \centering
        \includegraphics[scale=0.5]{theorems/transversal/Transversale_3.png}
    \end{figure}
  
  
  \item Les angles $\alpha$ et $\alpha'$ sont à l'extérieur du triangle $\triangle pp'r$. Dans ce cas-là, par l'isométrie de deux angles opposés par le sommet (sous-section 6.7), on se retrouve dans le même cas qu'en a). Donc $\alpha$ et $\alpha'$ ne peuvent pas être à l'extérieur du triangle $\triangle pp'r$.
  
  
  \begin{figure}[H]
    \centering
    \includegraphics[scale=0.7]{theorems/transversal/Transversale_2.PNG}
\end{figure}

  
 \end{enumerate}
 Le seul cas possible est donc le cas 2, $d$ et $d'$ sont parallèles.
\end{proof}

\begin{proof}
Nous considérons deux droites $d$ et $d'$ qui on $e$ comme transversale.

 \begin{figure}[H]
    \centering
    \includegraphics[scale=0.6]{theorems/transversal/Transversale_4.png}
\end{figure}


\begin{hyp}
     $d$, $d'$ et $e$ sont trois droites,
     $\alpha$ et $\alpha'$ sont correspondants et
     $d \parallel d'$
 \end{hyp}
 \begin{concl}
     $\alpha \equiv \alpha'$
 \end{concl}
 Nous construisons la droite $d''$ qui passe par $p$ en reportant l'angle $\alpha'$. Grâce à la première partie de la démonstration, nous déduisons que d' est parallèle à d''. Donc, comme d' et d'' sont parallèles et qu'elles croisent e au point p, ces deux droites sont confondues (axiome des parallèles). On en conclu que $\alpha \equiv \alpha'$.
\end{proof}

\end{document}
