\documentclass[a4paper,12pt]{article}

\input{packages}

\begin{document}

\pagebreak
\begin{corollary}
Deux triangles qui ont respectivement un côté et deux angles isométriques sont isométriques.
\end{corollary}

\begin{proof}
Considérons deux triangles $\triangle a_1b_1c_1$ et $\triangle a_2b_2c_2$.

\begin{figure}[H]
    \centering
    \includegraphics[scale=0.5]{theorems/corIsom2/corIsom2.png}
\end{figure}

\begin{hyp}
$a \equiv a'$, $\beta \equiv \beta'$ et $\alpha \equiv \alpha'$
\end{hyp}
\begin{concl}
$b \equiv b'$, $c \equiv c'$ et  $\gamma \equiv \gamma'$
\end{concl}

Pour démontrer ce corollaire, nous avons superposé ces deux triangles. Il existe alors trois cas possibles:
\begin{enumerate}
    \item $c_1 \equiv c_2$
    \item $c_1 > c_2$
    \item $c_1 < c_2$
\end{enumerate}
Dans le premier cas, comme $c_1 \equiv c_2$, $\beta_1 \equiv \beta_2$ et $a_1 \equiv a_2$, on sait que les deux triangles sont isométriques (axiome III).\\
Dans les deux autres cas, on peut observer la formation du triangle $b_1b_2'c_3$, dont l'un des angles est alpha 1. Considérons ce triangle, on peut observer que l'un de ses angles externes (voir sous-section 6.1) est $\alpha_2$, ce qui implique que $\alpha_1 < \alpha_2$, ce qui est absurde. Le seul cas possible est donc le premier.
\begin{figure}[H]
    \centering
    \includegraphics[scale=0.6]{theorems/corIsom2/corIsom2_2.png}
\end{figure}
\end{proof}

\begin{remark}
Un corollaire est une démonstration nouvelle que l'on peut faire grâce à une affirmation précédente. Le mot provient du latin "corollarium", qui signifie littéralement "petite couronne".\footnote{CNRTL, dictionnaire étymologique, http://www.cnrtl.fr/}
\end{remark}

\end{document}
