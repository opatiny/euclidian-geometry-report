\documentclass[a4paper,12pt]{article}

% don't forget the document class, generally : \documentclass[a4paper,12pt]{article}

\usepackage[utf8]{inputenc}
\usepackage[french]{babel}
\usepackage{graphicx}
\usepackage{gensymb}
\usepackage{amsmath}
\usepackage{float}
\usepackage{scrextend}
\usepackage{caption} 
\usepackage{siunitx}
\usepackage{enumitem}
\usepackage{amsthm}
\usepackage{fancyhdr}
\usepackage{amssymb}
\usepackage{wrapfig}
\usepackage{geometry}
\usepackage{standalone}
\usepackage{import}
\usepackage[usenames, dvipsnames]{color}

 \usepackage{biblatex} % manages bibliography and references
\addbibresource{sample.bib}


\geometry{hmargin=1in, vmargin=1in}

 \newenvironment{absolutelynopagebreak}
 {\par\nobreak\vfil\penalty0\vfilneg
 \vtop\bgroup}
 {\par\xdef\tpd{\the\prevdepth}\egroup
 \prevdepth=\tpd}
 
 \pagestyle{fancy}                        
\fancyhf{}                               
\fancyhf[HL]{Application des maths}                
\fancyhf[HR]{Géométrie euclidienne}             
\fancyhf[FC]{\thepage/\pageref{Lastpage}}
 
\newtheorem{definition}{Définition}[section]
\newtheorem{theorem}{Théorème}
\newtheorem{corollary}{Corollaire}[theorem]
\newtheorem{lemma}[theorem]{Lemme}
\newtheorem*{hyp}{Hypothèse}
\newtheorem*{concl}{Conclusion}
\newtheorem*{remark}{Remarque}

\captionsetup{format=default,labelformat=simple,labelsep=colon,
justification=justified,font={sf,small},labelfont=bf,
textfont=default} 



\begin{document}

\pagebreak
\subsubsection{Théorème 3}
\begin{theorem}
Théorème de Thalès: Toute droite parallèle à un des côtés d'un triangle et qui sectionne ses deux autres côtés forme un triangle diminué semblable au triangle initial.
\end{theorem}
\begin{remark}
Thalès a vécu au septième et sixième siècle avant J.C. dans la ville de Milet, en Grèce. Marchand de profession, il est aussi philosophe et l'un des précurseurs de la pensée scientifique moderne. En effet, au lieu de reposer l'explication de phénomènes inexpliqués sur la mythologie, il privilégia l'observation et son expérience personnelle. Ce procédé lui permit de faire de nombreuses découvertes et inventions d'une grande importance, notamment en astronomie, physique et mathématique.\\

On lui accorde, entre autre, la découverte d'un moyen de mesurer la hauteur des pyramides d'Egypte. Il observa qu'il existe chaque jour un moment lors duquel un objet et son ombre ont la même longueur. Par conséquent, lors d'une journée où les rayons du soleil sont perpendiculaires à un des côtés de la base de la pyramide (ce qui arrive deux fois par an), la hauteur de la pyramide sera égale à la longueur de son ombre à cette heure précise, à laquelle on additionne la moitié de la longueur du côté de la pyramide. Cette expérience est résumée sur le schéma ci-dessous (figure \ref{fig:thales}\footnote{Wikimedia, File:Thales Theorem 6.svg, Fred the Oyster, 20.11.2016}).
\footnote{Inspiré du site web Bibm@th.net, article "Thalès de Milet (624 av JC - 547 av JC)", 20.11.2016, http://www.bibmath.net/bios/index.php?action=affiche\&quoi=thales et de Wikipedia, article "Thalès", 20.11.2016, https://fr.wikipedia.org/wiki/Thal\%C3\%A8s\#cite\_note-61}


\begin{figure}[H]
    \centering
    \includegraphics[width= \textwidth]{theorems/semblables/thales/Thales_Theorem.eps}
    
    \caption{}
    \label{fig:thales}
\end{figure}
\end{remark}

\end{document}
