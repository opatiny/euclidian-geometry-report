\documentclass[a4paper,12pt]{article}

% don't forget the document class, generally : \documentclass[a4paper,12pt]{article}

\usepackage[utf8]{inputenc}
\usepackage[french]{babel}
\usepackage{graphicx}
\usepackage{gensymb}
\usepackage{amsmath}
\usepackage{float}
\usepackage{scrextend}
\usepackage{caption} 
\usepackage{siunitx}
\usepackage{enumitem}
\usepackage{amsthm}
\usepackage{fancyhdr}
\usepackage{amssymb}
\usepackage{wrapfig}
\usepackage{geometry}
\usepackage{standalone}
\usepackage{import}
\usepackage[usenames, dvipsnames]{color}

 \usepackage{biblatex} % manages bibliography and references
\addbibresource{sample.bib}


\geometry{hmargin=1in, vmargin=1in}

 \newenvironment{absolutelynopagebreak}
 {\par\nobreak\vfil\penalty0\vfilneg
 \vtop\bgroup}
 {\par\xdef\tpd{\the\prevdepth}\egroup
 \prevdepth=\tpd}
 
 \pagestyle{fancy}                        
\fancyhf{}                               
\fancyhf[HL]{Application des maths}                
\fancyhf[HR]{Géométrie euclidienne}             
\fancyhf[FC]{\thepage/\pageref{Lastpage}}
 
\newtheorem{definition}{Définition}[section]
\newtheorem{theorem}{Théorème}
\newtheorem{corollary}{Corollaire}[theorem]
\newtheorem{lemma}[theorem]{Lemme}
\newtheorem*{hyp}{Hypothèse}
\newtheorem*{concl}{Conclusion}
\newtheorem*{remark}{Remarque}

\captionsetup{format=default,labelformat=simple,labelsep=colon,
justification=justified,font={sf,small},labelfont=bf,
textfont=default} 



\begin{document}

\subsubsection{Théorème 1}
\begin{theorem}
Deux triangles dont les côtés sont respectivement parallèles ont des angles respectivement isométriques.
\end{theorem}

\begin{proof}
Nous considérons deux triangles quelconques $\triangle ABC$ et $\triangle A'B'C'$.

%shema triangles avec côtés parallèles 

\begin{hyp}
$a \parallel a'$, $b \parallel b'$ et $c \parallel c'$
\end{hyp}
\begin{concl}
$\alpha \equiv \alpha'$, $\beta \equiv \beta'$ et $\gamma \equiv \gamma'$
\end{concl}

Nous prolongeons $a$, $a'$ et $c'$. 

%shema triangles avec côtés prolongés (2)

Grâce au théorème de la transversale, nous observons que $\beta' \equiv \theta$, où theta est l'angle correspondant de beta.\\
Nous prolongeons $c$, $c'$ et $a$ et grâce au même théorème qu'auparavant, nous savons que $\beta \equiv \theta$.
Ainsi, $\beta \equiv \beta' \equiv \theta$.\\
Il suffit de répéter cette démonstration pour les deux autres angles des triangles afin d'obtenir $\alpha \equiv \alpha'$, $\beta \equiv \beta'$ et $\gamma \equiv \gamma'$.


\end{proof}

\end{document}