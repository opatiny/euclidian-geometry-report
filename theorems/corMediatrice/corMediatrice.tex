\documentclass[a4paper,12pt]{article}

\input{packages}

\begin{document}

\begin{corollary} \label{cor:mediatrices}
Les médiatrices d'un triangle se croisent en un point unique, le centre du cercle circonscrit du triangle $ABC$.
\end{corollary}
\begin{proof}
Nous considérons un triangle quelconque $ABC$. Nous nommons $I$ l'intersection des médiatrices des côtés $AC$ ($M_{AC}$) et $BC$ ($M_{BC}$).

\begin{hyp}
$I \in M_{AC}$ et $I \in M_{BC}$
\end{hyp}
\begin{concl}
$I \in M_{AB}$ (la médiatrice du côté $AB$) 
\end{concl}

En considérant l'hypothèse et la définition de la médiatrice (définition \ref{def:mediatrice}), nous savons que $IA \equiv IC$ et que $IB \equiv IC$. Par conséquent, $IA \equiv IB$. Puisque $I$ est à une distance équivalente de $A$ et de $B$, $I \in M_{AB}$.\\
De plus, puisque $I$ est le lieu des points équidistants à $A$, $B$ et $C$, c'est aussi le cercle du cercle circonscrit de $\triangle ABC$.
\end{proof}

\end{document}