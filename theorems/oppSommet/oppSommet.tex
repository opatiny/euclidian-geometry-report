\documentclass[a4paper,12pt]{article}

\input{packages}

\begin{document}

\pagebreak
\subsection{Isométrie de deux angles opposés par le sommet}
\begin{theorem}
Deux angles opposés par le sommet sont isométriques.
\end{theorem}
\begin{proof}
Nous considérons deux droites sécantes.

 \begin{figure}[H]
    \centering
    \includegraphics[scale=0.8]{schema/anglesym.PNG}
\end{figure}


\begin{hyp}
     Deux droites se croisent en un point
 \end{hyp}
 \begin{concl}
     $\alpha \equiv \beta$, $\gamma$ est l'angle complémentaire de $\alpha$ et $\beta$
 \end{concl}
 Comme $\gamma$ est l'angle complémentaire de $\alpha$ et $\beta$, on peut écrire l'équation suivante:
 \begin{equation}
 \gamma + \alpha = 180\degree \rightarrow \alpha = 180\degree-\gamma
 \end{equation}
 \begin{equation}
 \gamma+ \beta = 180\degree \rightarrow \beta = 180\degree-\gamma
 \end{equation}
 Par conséquent, on sait que $\alpha \equiv \beta$.
\end{proof}

\end{document}