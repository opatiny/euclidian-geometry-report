\documentclass[a4paper,12pt]{article}

\input{packages}

\begin{document}

\pagebreak
\subsection{Théorème de la somme des angles d'un triangle}
\begin{theorem}\label{th:180}
La somme des angles d'un triangle équivaut à 180 degrés.
\end{theorem}
\begin{proof}

Nous considérons un triangle $\triangle ABC$, puis nous traçons une parallèle au côté c passant par le sommet C.
\begin{figure}[H]
    \centering
    \includegraphics[scale=0.6]{theorems/sommeAngle/Somme_angles.PNG}
\end{figure}


\begin{hyp}
$\triangle ABC$ est quelconque
\end{hyp}
\begin{concl}
$\alpha + \beta +\gamma = 180 \degree$
\end{concl}
Grâce au théorème des transversales et à l'isométrie de deux angles opposés par le sommet, nous savons que $\alpha + \beta +\gamma = 180 \degree$.
\end{proof}

\end{document}
