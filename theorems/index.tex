\documentclass[a4paper,12pt]{article}

% don't forget the document class, generally : \documentclass[a4paper,12pt]{article}

\usepackage[utf8]{inputenc}
\usepackage[french]{babel}
\usepackage{graphicx}
\usepackage{gensymb}
\usepackage{amsmath}
\usepackage{float}
\usepackage{scrextend}
\usepackage{caption} 
\usepackage{siunitx}
\usepackage{enumitem}
\usepackage{amsthm}
\usepackage{fancyhdr}
\usepackage{amssymb}
\usepackage{wrapfig}
\usepackage{geometry}
\usepackage{standalone}
\usepackage{import}
\usepackage[usenames, dvipsnames]{color}

 \usepackage{biblatex} % manages bibliography and references
\addbibresource{sample.bib}


\geometry{hmargin=1in, vmargin=1in}

 \newenvironment{absolutelynopagebreak}
 {\par\nobreak\vfil\penalty0\vfilneg
 \vtop\bgroup}
 {\par\xdef\tpd{\the\prevdepth}\egroup
 \prevdepth=\tpd}
 
 \pagestyle{fancy}                        
\fancyhf{}                               
\fancyhf[HL]{Application des maths}                
\fancyhf[HR]{Géométrie euclidienne}             
\fancyhf[FC]{\thepage/\pageref{Lastpage}}
 
\newtheorem{definition}{Définition}[section]
\newtheorem{theorem}{Théorème}
\newtheorem{corollary}{Corollaire}[theorem]
\newtheorem{lemma}[theorem]{Lemme}
\newtheorem*{hyp}{Hypothèse}
\newtheorem*{concl}{Conclusion}
\newtheorem*{remark}{Remarque}

\captionsetup{format=default,labelformat=simple,labelsep=colon,
justification=justified,font={sf,small},labelfont=bf,
textfont=default} 



\begin{document}


\section{Théorèmes}
Dans cette partie, nous allons démontrer plusieurs théorèmes en ne se basant que sur ce que l'on sait déjà, c'est-à-dire les cinq axiomes.

\import{theorems/angle_externe/}{angle_externe}

\import{theorems/isom2/}{isom2}

\import{theorems/corIsom2/}{corIsom2}

\import{theorems/isocel/}{isocel}

\import{theorems/bissectrice/}{bissectrice}

\import{theorems/isom3/}{isom3}

\import{theorems/cosinus/}{cosinus}

\import{theorems/oppSommet/}{oppSommet}

\import{theorems/transversal/}{transversal}

\import{theorems/sommeAngle/}{sommeAngle}

\import{theorems/segmentMoyen/}{segmentMoyen}

\import{theorems/diagonalsParallelogram/}{text}

\import{theorems/mediatrice/}{mediatrice}
 
\import{theorems/corMediatrice/}{corMediatrice}

\import{theorems/hauteurs/}{hauteurs}

\import{theorems/semblables/}{semblables}
\import{theorems/semblables/th1/}{th1}
\import{theorems/semblables/th2/}{th2}
\import{theorems/semblables/thales/}{thales}

\import{theorems/mediane/}{mediane}

\import{theorems/casSimilitude/}{index}
\import{theorems/casSimilitude/cas1/}{cas1}
\import{theorems/casSimilitude/cas2/}{cas2}
\import{theorems/casSimilitude/cas3/}{cas3}

\end{document}

