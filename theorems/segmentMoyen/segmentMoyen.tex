\documentclass[a4paper,12pt]{article}

\input{packages}

\begin{document}

\pagebreak
\subsection{Théorème du segment moyen}
\begin{theorem}
Une droite passant par les milieux de deux côtés d'un triangle et parallèle au troisième côté de celui-ci.
\end{theorem}

\begin{proof}
Nous considérons le triangle $ABC$ et nous nommons $M$ et $N$ respectivement les milieux des côtés $a$ et $b$. 

\begin{hyp}
$AM \equiv MC$, $BM \equiv NC$
\end{hyp}

\begin{concl}
$AB \parallel MN$
\end{concl}

Nous commençons par construire un point $P$ sur le prolongement de $MN$ et pour que $PM \equiv MN$. On observe alors que $PANC$ est un parallélogramme, car par construction ses diagonales se coupent en leurs milieux ($PM \equiv MN$ et $AM \equiv MC$).\\
Ensuite, nous pouvons démontrer que le quadrilatère $PABN$ est aussi un parallélogramme, car $PA$ est parallèle et isométrique à $NB$. EN effet, $NB$ est le prolongement de $CN$, qui est parallèle et isométrique à $PA$, ainsi $PA \equiv CN \equiv NB$ (par hypothèse). Puisque les segments $PA$ et $NB$ sont isométriques et parallèles, le quadrilatère $PABN$ est un parallélogramme (propriété des parallèlogrammes) et comme ceux-ci ont comme particularité deux paires de côtés parallèles, l'on conclu que $MN \parallel AB$.


\end{proof}




\end{document}