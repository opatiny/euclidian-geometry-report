\documentclass[a4paper,12pt]{article}

\input{packages}

\begin{document}

\pagebreak
\subsection{Deuxième cas d'isométrie des triangles}
\begin{theorem}
Deux triangles qui ont respectivement un côté et les angles adjacents isométriques sont isométriques.
\end{theorem}
\begin{hyp}
$a_1 \equiv a_2$, $\beta_1 \equiv \beta_2$ et $\gamma_1 \equiv \gamma_2$
\end{hyp}
\begin{concl}
$b_1 \equiv b_2$, $c_1 \equiv c_2$ et que $\alpha_1 \equiv \alpha_2$
\end{concl}
\begin{proof}
Considérons deux triangles $\triangle a_1b_1c_1$ et $\triangle a_2b_2c_2$. Nous avons reporté le côté $c_2$ sur le côté $c_1$. alors, il y a trois possibilités:
\begin{figure}[H]
    \centering
    \includegraphics[scale=0.45]{theorems/isom2/cas2.png}
\end{figure}
\begin{enumerate}
\item $c_1 = c_2$, alors les deux triangles sont isométriques (axiome III)
\item $c_1<c_2$
\item $c_1>c_2$
\end{enumerate}

\pagebreak
Dans le deuxième et troisième cas, on forme le triangle $\triangle b_1b_3c_2'$. Ce triangle est isométrique au triangle $\triangle a_2b_2c_2$, car ils ont en commun un angle compris entre deux côtés isométriques. Par conséquent, $\gamma_2\equiv \gamma_3$, donc $\gamma_1$ et $\gamma_3$ sont confondus, d est nul et $c_1\equiv c_2$. On en revient donc au premier cas: les deux triangles $\triangle a_1b_1c_1$ et $\triangle a_2b_2c_2$ sont isométriques grâce au premier cas d'isométrie des triangles.


\begin{figure}[H]
    \centering
    \includegraphics[scale=0.6]{theorems/isom2/Cas2_2.png}
\end{figure}



\end{proof}

\end{document}
