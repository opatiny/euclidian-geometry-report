\documentclass[a4paper,12pt]{article}

\input{packages}

\begin{document}

\pagebreak
\subsubsection{Théorème de l'intersection des médianes}
Commençons par définir la médiane.
\begin{definition}{Médiane:}
Droite passant par le sommet d'un triangle et qui partage le côté opposé en deux parties égales.
\end{definition}

\begin{theorem}
Les médianes d'un triangle concourent en un point situé à deux tiers des sommets du triangle.
\end{theorem}

\begin{proof}
Nous considérons le triangle quelconque $ABC$ et nous représentons les point A', B' et C' qui coupent les respectivement les côtés a, b et c en leurs milieux. 
\begin{hyp}
Les segments AA' et BB' sont confondus avec les médianes du triangle ABC et ils se croisent en un point G.
\end{hyp}
\begin{concl}
La droite qui passe par les points C et G coupe le segment AB en C'.
\end{concl}

\end{proof}
\end{document}