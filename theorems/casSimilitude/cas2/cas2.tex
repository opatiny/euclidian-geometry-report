\documentclass[a4paper,12pt]{article}

% don't forget the document class, generally : \documentclass[a4paper,12pt]{article}

\usepackage[utf8]{inputenc}
\usepackage[french]{babel}
\usepackage{graphicx}
\usepackage{gensymb}
\usepackage{amsmath}
\usepackage{float}
\usepackage{scrextend}
\usepackage{caption} 
\usepackage{siunitx}
\usepackage{enumitem}
\usepackage{amsthm}
\usepackage{fancyhdr}
\usepackage{amssymb}
\usepackage{wrapfig}
\usepackage{geometry}
\usepackage{standalone}
\usepackage{import}
\usepackage[usenames, dvipsnames]{color}

 \usepackage{biblatex} % manages bibliography and references
\addbibresource{sample.bib}


\geometry{hmargin=1in, vmargin=1in}

 \newenvironment{absolutelynopagebreak}
 {\par\nobreak\vfil\penalty0\vfilneg
 \vtop\bgroup}
 {\par\xdef\tpd{\the\prevdepth}\egroup
 \prevdepth=\tpd}
 
 \pagestyle{fancy}                        
\fancyhf{}                               
\fancyhf[HL]{Application des maths}                
\fancyhf[HR]{Géométrie euclidienne}             
\fancyhf[FC]{\thepage/\pageref{Lastpage}}
 
\newtheorem{definition}{Définition}[section]
\newtheorem{theorem}{Théorème}
\newtheorem{corollary}{Corollaire}[theorem]
\newtheorem{lemma}[theorem]{Lemme}
\newtheorem*{hyp}{Hypothèse}
\newtheorem*{concl}{Conclusion}
\newtheorem*{remark}{Remarque}

\captionsetup{format=default,labelformat=simple,labelsep=colon,
justification=justified,font={sf,small},labelfont=bf,
textfont=default} 



\begin{document}

\pagebreak
\subsubsection{Théorème de l'intersection des médianes}
Commençons par définir la médiane.
\begin{definition}{Médiane:}
Droite passant par le sommet d'un triangle et qui partage le côté opposé en deux parties égales.
\end{definition}

\begin{theorem}
Les médianes d'un triangle concourent en un point situé à deux tiers des sommets du triangle.
\end{theorem}

\begin{proof}
Nous considérons le triangle quelconque $ABC$ et nous représentons les point A', B' et C' qui coupent les respectivement les côtés a, b et c en leurs milieux. 
\begin{hyp}
Les segments AA' et BB' sont confondus avec les médianes du triangle ABC et ils se croisent en un point G.
\end{hyp}
\begin{concl}
La droite qui passe par les points C et G coupe le segment AB en C'.
\end{concl}

\end{proof}
\end{document}