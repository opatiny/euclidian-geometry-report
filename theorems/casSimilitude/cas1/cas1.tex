\documentclass[a4paper,12pt]{article}

\input{packages}

\begin{document}

\subsubsection{Premier cas de similitude}
\begin{theorem}
Deux triangles dont les angles sont respectivement isométriques sont semblables.
\end{theorem}

\begin{remark}
Si deux triangles ont deux angles respectivement isométriques, leur dernier angle est aussi isométrique, car la somme des angles d'un triangle est une constante (théorème \ref{th:180}). Imaginons $\triangle ABC$ et $\triangle A'B'C'$. Si nous considérons $\alpha \equiv \alpha'$ et $\beta \equiv \beta'$, alors $\alpha + \beta = 180 - \gamma$ et $\alpha' + \beta' = 180 - \gamma'$, donc $\gamma \equiv \gamma'$.

\end{remark}


\begin{proof}
Nous considérons deux triangles $\triangle ABC$ et $\triangle A'B'C'$.

\begin{hyp}
$\alpha \equiv \alpha'$ et $\beta \equiv \beta'$
\end{hyp}

\begin{concl}
$\frac{a}{a'} = \frac{b}{b'} = \frac{c}{c'}$
\end{concl}




\end{proof}
\end{document}