\documentclass[a4paper,12pt]{article}

\input{packages}

\begin{document}

\pagebreak
\subsection{Troisième cas d'isométrie des triangles}
\begin{theorem}
Deux triangles qui ont respectivement les trois côtés isométriques sont isométriques.
\end{theorem}
\begin{proof}
Nous considérons deux triangles $a_1b_1c_1$ et $a_2b_2c_2$.

\begin{figure}[H]
    \centering
    \includegraphics[scale=0.6]{theorems/isom3/Cas_3.PNG}
\end{figure}

 \begin{hyp}
     $a_1\equiv a_2$,
     $b_1\equiv b_2$ et
     $c_1\equiv c_2$
 \end{hyp}
 \begin{concl}
     $\alpha \equiv \alpha'$,
     $\beta \equiv \beta'$ et
     $\gamma \equiv \gamma'$
 \end{concl}
 Nous rassemblons les deux triangles en un quadrilatère en superposant $c_1$ et $c_2$. 
 
 \begin{figure}[H]
    \centering
    \includegraphics[scale=0.6]{theorems/isom3/Cas3_2.png}
\end{figure}
 
 Puis nous relions les sommets en $\gamma$ et $\gamma'$, nous nommons ce segment $h$. Ainsi, nous obtenons deux triangles qui sont isocèles $\triangle a_1a_2h$ et $\triangle b_1b_2h$ ($a_1\equiv a_2$ pour $\triangle a_1a_2h$ et $b_1\equiv b_2$ pour $\triangle b_1b_2h$, voir section 6.3) ce qui signifie que $\theta_1 \equiv \theta_2$ et que $\theta_3 \equiv \theta_4$.
 
 
 \begin{figure}[H]
    \centering
    \includegraphics[scale=0.6]{theorems/isom3/Cas3_3.png}
\end{figure}

 
 
 Finalement, comme $\alpha \equiv \theta_1 + \theta_2$ et $\alpha' \equiv \theta_3 + \theta_4$, nous avons démontré que $\alpha \equiv \alpha'$. Les deux triangles $a_1b_1c_1$ et $a_2b_2c_2$ sont donc isométriques (axiome III).
\end{proof}

\end{document}
