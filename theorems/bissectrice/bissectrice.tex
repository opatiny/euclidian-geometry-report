\documentclass[a4paper,12pt]{article}

\input{packages}

\begin{document}


\pagebreak
\subsection{Propriétés des bissectrices}
\begin{theorem}
La bissectrice d'un angle est le lieu géométrique des points intérieurs à l'angle et équidistants à ses côtés.
\end{theorem}
Le théorème implique deux choses:
\begin{enumerate}
\item Un point qui se trouve sur la bissectrice d'un angle est équidistant à ses côtés
\item Un point équidistant au côtés d'un angle se trouve sur sa bissectrice
\end{enumerate}
\begin{proof}
Nous considérons deux droites $d$ et $d'$ qui se coupent en un point $A$, on trace la bissectrice de l'angle en $A$ et on obtient deux angles $\alpha$ et $\alpha'$. Puis nous traçons deux segments $a$ et $a'$ qui partent d'un point $Q$ sur la bissectrice et qui coupent les droites $d$ et $d'$ perpendiculairement, aux points $D$ et $D'$.
\begin{figure}[H]
    \centering
    \includegraphics[scale=0.8]{schema/Bissectrices_1.PNG}
\end{figure}


\begin{hyp}
$d$ et $d'$sont deux droites sécantes, $\alpha \equiv \alpha'$, $a\perp d$, $a' \perp d'$
\end{hyp}
\begin{concl}
$a\equiv a'$
\end{concl}
En faisant la construction, nous obtenons deux triangles $\triangle AQD$ et $\triangle A'QD'$. Comme ces deux triangles partagent un côté et ont deux angles isométriques ($\alpha \equiv \alpha'$, les angles droits), ils sont isométriques (corollaire du deuxième cas d'isométrie des triangles). Donc $a \equiv a'$.
\end{proof}

\begin{proof}
Nous considérons deux droites $d$ et $d'$ qui se coupent en un point $A$. Puis nous considérons un point $Q$ qui est à égale distance de $d$ et $d'$. Ensuite nous traçons deux segments $a$ et $a'$ qui coupent les droites $d$ et $d'$ perpendiculairement aux points $D$ et $D'$ et qui passent par $Q$. En reliant le point Q et le point A, nous obtenons deux angles $\alpha$ et $\alpha'$.

 \begin{figure}[H]
    \centering
    \includegraphics[scale=0.8]{schema/Bissectrices_3.PNG}
\end{figure}


\begin{hyp}
$d$ et $d'$sont deux droites sécantes, $a\equiv a'$, $a\perp d$, $a' \perp d'$
\end{hyp}
\begin{concl}
$\alpha \equiv \alpha'$
\end{concl}
Ainsi, nous obtenons deux triangle $\triangle AQD$ et $\triangle A'QD'$. Pour le moment, nous savons que ces deux triangles ont un côté en commun et un angle isométrique. Pour trouver une troisième grandeur isométrique, nous formons le triangle $\triangle DD'Q$. Ce triangle est isocèle, parce que $a\equiv a'$. Donc le triangle $\triangle DD'A$ l'est aussi et les distances $AD'$ et $AD$ sont isométriques. 

 \begin{figure}[H]
    \centering
    \includegraphics[scale=0.8]{schema/Bissecrtices_4.PNG}
\end{figure}


Donc, grâce au corollaire du deuxième cas d'isométrie, nous savons que les triangle $\triangle AQD$ et $\triangle A'QD'$ sont isométriques. Par conséquent, $\alpha \equiv \alpha'$.
\end{proof}

\end{document}